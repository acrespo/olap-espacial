\documentclass[a4paper,12pt]{article}
\usepackage[utf8]{inputenc}

\begin{document}

\title{Trabajo Práctico Especial - OLAP}

\author{
{\rm Alvaro Crespo}
\and
{\rm Esteban Ordano}
}
\date{}

\maketitle

\section{Introducción}

Este repositorio contiene el código y los scripts necesarios para
instalar y ejecutar dos funciones de agregación: \texttt{ST\_Intersects}
y \texttt{ST\_NearCentroid}, como fue descripto en el enunciado
(disponible en el repositorio como \texttt{enunciado.pdf}).

\section{Instalación}

\subsection{Ubuntu 12.04 en adelante}

\begin{enumerate}
\def\labelenumi{\arabic{enumi}.}
\item
  Instalar el motor de bases de datos postgresql y la extensión PostGIS,
  que añade funcionalidad para trabajar con datos geoespaciales.

\begin{verbatim}
sudo apt-get install python-software-properties
sudo apt-add-repository ppa:ubuntugis/ppa
sudo apt-get update
sudo apt-get install postgresql-9.1
sudo apt-get install postgresql-9.1-postgis
\end{verbatim}

  A partir de este momento, ejecutar los comandos como usuario con
  permisos para utilizar la base de datos (en Ubuntu 12.04 este usuario
  se llama \texttt{postgres} por defecto).

\begin{verbatim}
sudo su postgres
\end{verbatim}

\item
  Crear una base de datos utilizando \texttt{createdb}. Por ejemplo:

\begin{verbatim}
createdb olap
\end{verbatim}
\item
  Ejecutar \texttt{install.sh} con el nombre de la base de datos como
  argumento. Ejemplo:

\begin{verbatim}
./install.sh olap
\end{verbatim}
\item
  (Opcional) Ejecutar los tests para verificar la correcta instalación
  de las funciones:

\begin{verbatim}
./run_tests.sh olap
\end{verbatim}
\end{enumerate}
\subsection{Windows 64 bits}

\begin{enumerate}
\def\labelenumi{\arabic{enumi}.}
\item
  Instalar Postgres y Postgis:

  \begin{enumerate}
  \def\labelenumii{\arabic{enumii}.}
  \itemsep1pt\parskip0pt\parsep0pt
  \item
    PostgreSQL:
    {[}http://get.enterprisedb.com/postgresql/postgresql-9.2.4-1-windows-x64.exe{]}
  \item
    Postgis:
    {[}http://download.osgeo.org/postgis/windows/pg92/postgis-pg92x64-setup-2.0.3-2.exe{]}
  \end{enumerate}
\item
  Crear una base de datos con posibilidad para usar PostGIS:

  \begin{enumerate}
  \def\labelenumii{\arabic{enumii}.}
  \itemsep1pt\parskip0pt\parsep0pt
  \item
    Ejecutar pgAdminIII
  \item
    Utilizar pgAdminIII para crear una nueva base de datos, por ejemplo,
    \texttt{olap}.
  \item
    Estando pgAdminIII conectado a esa base de datos, ejecutar el
    archivo \texttt{install.sql} que se encuentra en esta carpeta.
  \end{enumerate}
\end{enumerate}

\section{Ejecución de pruebas}

\subsection{*nix}

\begin{enumerate}
\def\labelenumi{\arabic{enumi}.}
\itemsep1pt\parskip0pt\parsep0pt
\item
  Ejecutar el script de bash \texttt{run\_tests.sh}.
\end{enumerate}

\subsection{Windows}

\begin{enumerate}
\def\labelenumi{\arabic{enumi}.}
\item
  Ejecutar el script \texttt{test\_st\_intersects.sql} desde pgAdminIII
  para testear la función de agregación \texttt{ST\_Intersects}.
\item
  Ejecutar el script \texttt{test\_st\_nearcentroid.sql} desde
  pgAdminIII para testear la función de agregación
  \texttt{ST\_NearCentroid}.
\end{enumerate}

\end{document}
